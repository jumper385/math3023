\documentclass[a4paper, 11pt, twoside]{article}
\usepackage[margin=1in]{geometry}
\usepackage{amsmath}
\usepackage{amssymb}
\usepackage{graphicx}
\usepackage{inputenc}

\title{
    \large \textbf{MATH3023 - Lecture 1} \\
    Revision \& Pre-Requesite Knowledge
}
\author{Henry Chen}

\begin{document}
\maketitle
\tableofcontents

\section{Integration}
\emph{All integrals in this unit will be solvable!}
We treat integrals like ``ares'' under the curve.
\subsection{The Fundamental Theorem of Calculus}
If the function $f$ is continuous between $[a,b]$, then $F$ is an \emph{antiderivative} of $f$.
\subsection{Indefinite Integrals}
We are looking for the most \emph{general} antiderivative. They will always have $+C$ at the end of the function. 

\section{Techniques of Integration}
\subsection{Integration by Simple Substitution}
Make the integral look like something on the formula sheet. 
\subsection{Integration by Trigonometric Substitution}
If the simple substitution fails, try to turn everything into $sin$ and $cos$. This involves using trig functions. An example of this is below:
\[
    \int{\sin^{m}{x}\cos^{n}{x}} \,\mathrm{d}x
\]
If either $m$, $n$ or both are odd, we peel off one of the odd powers and use the trigonometric identity $\sin^{2}{x} + \cos^2{x} = 1$ \emph{and} use $u=\sin{x}$ or $u=\cos{x}$ to the peeled off powers.

If either $m$, $n$ or both are even, we use the half-angle formula $\cos(2x) = \cos^{2}{x} - 1 = 1 - 2\sin^{2}{x} $ for either $\sin^{2}{x}$ or $\cos^{2}{x}$.

\[
    \sin^{2}{x} = \frac{1}{2}[1-\cos{2x}] \, , \, \cos^{2}{x} = \frac{1}{2}[1+\cos{2x}]
\]
\subsection{Integration by Inverse Trigonometric Substitution}
This is used to get rid of the square root of a substitution and getting rid of it is impossible with a simple substitution. We rely on a reference right angle triangle to achieve this technique. All formulae are on the formula sheet.

\end{document}